% This is samplepaper.tex, a sample chapter demonstrating the
% LLNCS macro package for Springer Computer Science proceedings;
% Version 2.20 of 2017/10/04
%
\documentclass[runningheads]{llncs}

\usepackage{graphicx}
% Used for displaying a sample figure. If possible, figure files should
% be included in EPS format.
%
% If you use the hyperref package, please uncomment the following line
% to display URLs in blue roman font according to Springer's eBook style:
% \renewcommand\UrlFont{\color{blue}\rmfamily}

\begin{document}
%
\title{Understanding the trade-offs of causal replication solutions through simulation}
%
%\titlerunning{Abbreviated paper title}
% If the paper title is too long for the running head, you can set
% an abbreviated paper title here
%
\author{António Nunes Duarte \and
João Leitão \and
Pedro Fouto}
%
\authorrunning{F. Author et al.}
% First names are abbreviated in the running head.
% If there are more than two authors, 'et al.' is used.
%
\institute{NOVA School of Science and Technology}
%
\maketitle              % typeset the header of the contribution
%
\begin{abstract}
Causal consistency is a weak consistency model, that works by 
tracking causal dependencies between operations and making sure that they are propagated 
and displayed, to users of the system, in a causally consistent order.
Many different ways of achieving causal consistency have been proposed in the literature
over the years, (eg. usage of Logical Clocks\cite{baquero2016logical}, 
Scalar Timestamps\cite{du2014gentlerain}, tracking Direct Dependencies\cite{lloyd2011don}, 
usage of specific node topologies that naturally ensure consistency\cite{van2020intrinsic}),
but each comes with it's own set of trade-offs which turn the decision complicated 
and	highly dependent on the specification of the system that one is looking to develop. The aim of this work, is 
exploring those trade-offs through the usage of a simulator, to more thoroughly understand the limits,
and optimal use cases of each proposed solution.

\keywords{Causality Tracking \and Consistency \and Simulator}
\end{abstract}

\section{Introduction}

\section{Related Work}

\section{Causality Tracking}

\section{Progress Report}

%
% ---- Bibliography ----
%
% BibTeX users should specify bibliography style 'splncs04'.
% References will then be sorted and formatted in the correct style.
%
% \bibliographystyle{splncs04}
% \bibliography{mybibliography}
%

\bibliographystyle{splncs04}
\bibliography{references}

\end{document}

